\documentclass[11pt]{article}
\usepackage[margin=2.0cm,nohead]{geometry}
\usepackage{url}
\usepackage{graphicx}
\usepackage{tabularx}

\title{Project 2 \\ Advanced Database Systems \\}

\author{
Amandeep Singh\\as3947@columbia.edu
\and
Evangelia Sitaridi \\ es2996@cs.columbia.edu
}

\date{\today}


\begin{document}
\maketitle


\section{Dataset}
This dataset \cite{911d} describes the data of interviews of World Trade Center 911 workers. We found it very interesting since it combined various demographic attributes and attributes
describing the consequences from the events of 911. The attributes of the dataset are described in table \ref{911t} and their name in their initial dataset is in parentheses.

\begin{table}
\begin{tabular}{|l|l|p{5cm}|}
\hline
VARIABLE\_NAME & VARIABLE\_LABEL & QUESTION ASKED? \\
\hline
GENDER & Gender & What is the enrollee's sex? \\
\hline
EMPLOYED\_911 &
Employment Status on 9/11/01 &
Was the enrollee employed on 9/11/01?\\
\hline
SMOKER\_ST &
Smoking Status &
What was the smoking status of the enrollee at the time of interview?\\
\hline
COUGH &
New or Worsening Cough &
Did the enrollee report new onset or worsening of a persistent cough since 9/11/01?\\
\hline
DEPRESSION &
New or Worsening Depression &
Did the enrollee report new onset or worsening depression, anxiety, or other emotional problems since 9/11/01?\\
\hline
HEADACHE &
New or Worsening Headache &
Did the enrollee report new onset or worsening frequent, severe headaches since 9/11/01?\\
\hline
BREATH &
New or Worsening Shortness of Breath &
Did the enrollee report new onset or worsening of shortness of breath since 9/11/01?\\
\hline
SKIN &
New or Worsening Skin Problems &
Did the enrollee report new onset or worsening of skin rash/irritation since 9/11/01?\\
\hline
THROAT &
New or Worsening Throat Irritation &
Did the enrollee report new onset or worsening of throat irritation since 9/11/01?\\
\hline
DUST &
Dust &
Did the enrollee indicate that he/she was outdoors on 9/11/01 within the dust or debris cloud resulting from the collapse of the World Trade Center?\\
\hline
EDUCATION &
Education Level &
What was the highest grade or year of school completed by the enrollee at the time of interview?\\
\hline
INCOME\_SLAB &
Household Income &
What was the enrollee's total household income in 2002 before taxes?\\
\hline
AGE\_GROUP &
Age Group on 9/11/01 &
What was the enrollee's age on 9/11/01?\\
\hline
CENSUS\_RACE &
Race/Ethnicity &
What is the enrollee's combined race/ethnicity?\\
\hline
MARITAL\_STATUS &
Marital\ Status &
What was the enrollee's marital status at the time of interview?\\
\hline
\end{tabular}
\caption{911 Dataset Attributes}
\label{911t}
\end{table}

\section{Data Preprocessing}

The initial dataset [1] contained 71427 rows and more than fifty columns. Thus the initial size was too large both in terms of rows and column number. To reduce the execution time of the program and facilitate the testing of our code we did the following modifications:
\begin{enumerate}
   \item keep the first 1000 rows of dataset
   \item select 17 columns we found most interesting, and are the ones described later. We first removed the columns that had mostly NULL values. Some of the columns were removed after      experimentation since they did not add to the results.
   \item due to a problem in the format of the initial dataset the value of the participants' income was represented using comma, e.g.
   \item attibute renamed to more intuitive ones
25,000 instead of 25.000. This confused our program since it regarded as an extra column. That is why we replaced each  \textit{,000} with string  \textit{000}.
\end{enumerate}
Finally, the quotes around the value of each column were added by OpenOffice SpreadSheet program but we let them since we did not feel they affected the legibility of the results.

\section{Implementation}

The dataset is stored in the DataSet class object, each row in the dataset is defined as a Transaction class object. If a dataset has a title in the first line, we add as an item value
the name of the attribues concatenated with the value of the attribute.

\begin{verbatim}
Algorithm:

DataSet(fileName, hasTitle)
    if(hasTitle)
       title_attrs=title.split(",")
    foreach (line)
        line_attrs = line.split(",")
        if(hasTitle)
            // adds item: title_attrs[i] = line_attr[i]  in the list
            transactions.add(line_attrs,title_attrs)
        else
            transactions.add(line_attrs)
\end{verbatim}

After each iteration of the Apriori algorithm we generate a new instance of ItemSet class.
So, for the first round of the algorithm we use.

\begin{verbatim}
GenerateL1()
    singles // stores the frequency of each item.
    itemsL1 // stores the items which have support > min_support
    
    transactions = dataset.getTransactions();
    foreach (transactions)
         foreach (item in transactions)
              calculate the frequency for each item.
    
    foreach (singles)
        supp=singles.get(s) / (dataset.size() * 1.0)
        if (supp >= minSupp) 
            itemsL1 // add a new ItemSet(singles)
\end{verbatim}

Subsequent rounds of Apriori uses this base itemsL1, new string is added into the item set using nextString(ItemSet) method.

\begin{verbatim}
nextString(ItemSet i1, ItemSet i2)
    // as the item set is generated iteratively, length of i1 & i2 should match
    if len(i1) != len(i2)
        return null        
    else 
        len_x = length of i1
    // check both the strings are equal except the last element
    while (len_x > 1)
         if i1.item() != i2.item()
              return null
         // check the last element
         x = last element of i1
         y = last element of i2
         if (x < y)
              return y
    return null
\end{verbatim}


\section{Results}

During our data analysis we faced the problem of many complex rules  so apart from saving the frequent item-sets and rules we create an extra file: top\_output.txt which contains only
top five-rules for each possible conclusion.  Below we analyze the most interesting associations in the selected dataset.

\textbf{Support 10\% Confidence 60\%}

\textit{Rule 1}

["COUGH"="No", "EMPLOYED\_911"="Yes", "INCOME\_SLAB"="\$150000 or More", 

"SKIN"="No", "THROAT"="No"] \\ 
$=>$ ["EDUCATION"="College +"] Conf: 95.24\% , Supp: 12.01\%  

This rules associates education of the participants to income, smoking and hearing condition after 911 events. It seems that people with high income, employed and
relatively not physically affected 911 events are educated. \\

\textit{Rule 2}

["COUGH"="No", "GENDER"="Male", "INCOME\_SLAB"="\$150000 or More", "SKIN"="No"] \\
$=>$ ["CENSUS\_RACE"="White (non Hispanic)"] Conf: 92.66 \% , Supp: 10.11 \% 

Again, there is another demographic that weatlhy males were white. \\

\textit{Rule 3}

["AGE\_GROUP"="25-44 years", "CENSUS\_RACE"="White (non Hispanic)", "DEPRESSION"="No", "DUST"="No"]\\
 $=>$ ["EMPLOYED\_911"="Yes"] Conf: 98.26 \% , Supp: 11.31 \%  \\

\textit{Rule 4}

["COUGH"="Yes", "THROAT"="Yes"] $=>$ ["BREATH"="Yes"] Conf: 64.50\% , Supp: 12.91\%  \\

This rules correlates different conditions caused by 911 events. It seems that participants experiencing cough and throat irritation also experience shortness of breath.\\

\textit{Rule 5}

["AGE\_GROUP"="25-44 years", "COUGH"="No", "DEPRESSION"="No", "EDUCATION"="College +", "HEADACHE"="No", "THROAT"="No"]\\
$=>$ ["SMOKER\_ST"="Never smoked"] Conf: 72.60\% , Supp: 10.61\%  \\

\textit{Rule 6}

["BREATH"="Yes", "THROAT"="Yes"] =$>$["DEPRESSION"="Yes"] Conf: 69.41\% , Supp: 11.81\%  \\

This rules relates a mental symptom to physical symptoms. People physically affected from 911 events likely experience symptoms
of depression. \\

\textit{Rule 7}

["DUST"="Yes", "THROAT"="Yes"] =$>$["DEPRESSION"="Yes"] Conf: 74.64\% , Supp: 10.31\%  \\

\textit{Rule 8}

["DEPRESSION"="Yes"] =$>$["GENDER"="Female"] Conf: 44.62\% , Supp: 20.32\%  

["DEPRESSION"="Yes"] =$>$["GENDER"="Male"] Conf: 55.38\% , Supp: 25.23\% 

This rule implies that depression is more prevalent to male participants. \\

\textit{Rule 9}

["SMOKER\_ST"="Current smoker"] =$>$["EDUCATION"="College +"] Conf: 63.92\% , Supp: 10.11\%  \\

\textbf{Support 5\% Confidence 35\%}

We decided to focus on underpresented groups of participants which might be covered by the previous setting of the algorithm

\textit{Rule 10}

["CENSUS\_RACE"="White (non Hispanic)", "GENDER"="Female", "THROAT"="Yes"] \\
$=>$["DEPRESSION"="Yes"] Conf: 81.48\% , Supp: 6.61\%   \\

["CENSUS\_RACE"="White (non Hispanic)", "EMPLOYED\_911"="Yes", "GENDER"="Female", "THROAT"="Yes"]\\
 $=>$["DEPRESSION"="Yes"] Conf: 82.86\% , Supp: 5.81\%  \\

The second of the two rules above sets an extra condition, that a woman is also employed and has higher confidence. We can conclude for that depression
is slightly more common among employed women. \\

\textit{Rule 11}

["BREATH"="Yes", "HEADACHE"="Yes", "THROAT"="Yes"] $=>$ ["DEPRESSION"="Yes"] Conf: 83.87\% , Supp: 5.21\%  \\

Remembering Rule 6 adding headache condition the chance for a person to be depressed increases significantly from 69.41\% to 83.87\%. \\

\textit{Rule 12}

["AGE\_GROUP"="25-44 years", "INCOME\_SLAB"="\$25000 to less than \$75000"] $=>$["MARITAL\_STATUS"="Not Married"] Conf: 52.73\% , Supp: 5.81\% 

This rules gives us again information on the demographic profile of the participants. Participants aged 25-44 having average income is more likely
not to be married.\\

\textit{Rule 13}

["DEPRESSION"="No", "INCOME\_SLAB"="\$150000 or More", "MARITAL\_STATUS"="Married", "SMOKER\_ST"="Never smoked"] \\
$=>$["EDUCATION"="College +"] Conf: 100.00\% , Supp: 5.11\% 


\textit{Rule 14}
Top Reasons for depression:

["BREATH"="Yes", "HEADACHE"="Yes", "THROAT"="Yes"] =>["DEPRESSION"="Yes"] Conf: 83.87\% , Supp: 5.21\% 
["CENSUS_RACE"="White (non Hispanic)", "EMPLOYED_911"="Yes", "GENDER"="Female", "THROAT"="Yes"] =>["DEPRESSION"="Yes"] Conf: 82.86\% , Supp: 5.81\% 
["SKIN"="Yes", "THROAT"="Yes"] =>["DEPRESSION"="Yes"] Conf: 82.35\% , Supp: 5.61\% 
["BREATH"="Yes", "EMPLOYED_911"="Yes", "HEADACHE"="Yes"] =>["DEPRESSION"="Yes"] Conf: 82.05\% , Supp: 6.41\% 
["CENSUS_RACE"="White (non Hispanic)", "GENDER"="Female", "THROAT"="Yes"] =>["DEPRESSION"="Yes"] Conf: 81.48\% , Supp: 6.61\% 

\section* {Appendix}

\subsection* {File Listing}

We list below the files included in our submission. The compressed file contains the following sub-folders: [dataset,src].

The files are:
\begin{itemize}
\item README.pdf
\item src/APriori.java
\item src/ItemSet.java
\item src/DataSet.java
\item src/Transaction.java
\item src/Makefile
\item dataset/INTEGRATED-DATASET.csv
\item output/output.txt
\item output/top\_output\_.txt
\end{itemize}

\subsection* {Compilation}
\begin{itemize}
\item \textbf{Compile}: make
\item \textbf{Clean-up}: make clean (removes .class files and output files)
\item\textbf{Execute}: java APriori  $min\_supp$ $min\_conf$ (from src directory)

\end{itemize}


\begin{thebibliography}{50}
\bibitem{911d} \textit{911 Dataset}, \url{http://www.nyc.gov/html/datamine/html/data/terms.html?dataSetJs=raw.js&theIndex=22}
\end{thebibliography}


\end{document}